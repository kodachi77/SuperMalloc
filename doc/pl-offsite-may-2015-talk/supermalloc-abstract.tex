SuperMalloc is an implementation of \texttt{malloc(3)} originally
designed for X86 Hardware Transactional Memory (HTM)\@.  It turns out
that the same design decisions also make it fast even without HTM\@.
For the malloc-test benchmark, which is one of the most difficult
workloads for an allocator, with one thread SuperMalloc is about 2.1
times faster than the best of DLmalloc, JEmalloc, Hoard, and
TBBmalloc; with 8 threads and HTM, SuperMalloc is 2.75 times faster;
and on 32 threads without HTM SuperMalloc is 3.4 times faster.
SuperMalloc generally compares favorably with the other allocators on
speed, scalability, speed variance, memory footprint, and code size.

SuperMalloc achieves these performance advantages using less than half
as much code as the alternatives.  SuperMalloc exploits the fact that
although physical memory is always precious, virtual address space on
a 64-bit machine is relatively cheap.  It allocates
\unit{2}\mebi\byte{} chunks which contain objects all the same size.
To translate chunk numbers to chunk metadata, SuperMalloc uses a
simple array (most of which is uncommitted to physical memory).
SuperMalloc takes care to avoid associativity conflicts in the cache:
most of the size classes are a prime number of cache lines, and
nonaligned huge accesses are randomly aligned within a page.  Objects
are allocated from the fullest non-full page in the appropriate size
class.  For each size class, SuperMalloc employs a 10-object
per-thread cache, a per-CPU cache that holds about a level-2-cache
worth of objects per size class, and a global cache that is organized
to allow the movement of many objects between a per-CPU cache and the
global cache using $O(1)$ instructions.  SuperMalloc prefetches
everything it can before starting a critical section, which makes the
critical sections run fast, and for HTM improves the odds that the
transaction will commit.
